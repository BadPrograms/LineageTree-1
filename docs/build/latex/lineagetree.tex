%% Generated by Sphinx.
\def\sphinxdocclass{report}
\documentclass[letterpaper,10pt,english]{sphinxmanual}
\ifdefined\pdfpxdimen
   \let\sphinxpxdimen\pdfpxdimen\else\newdimen\sphinxpxdimen
\fi \sphinxpxdimen=.75bp\relax

\PassOptionsToPackage{warn}{textcomp}
\usepackage[utf8]{inputenc}
\ifdefined\DeclareUnicodeCharacter
% support both utf8 and utf8x syntaxes
  \ifdefined\DeclareUnicodeCharacterAsOptional
    \def\sphinxDUC#1{\DeclareUnicodeCharacter{"#1}}
  \else
    \let\sphinxDUC\DeclareUnicodeCharacter
  \fi
  \sphinxDUC{00A0}{\nobreakspace}
  \sphinxDUC{2500}{\sphinxunichar{2500}}
  \sphinxDUC{2502}{\sphinxunichar{2502}}
  \sphinxDUC{2514}{\sphinxunichar{2514}}
  \sphinxDUC{251C}{\sphinxunichar{251C}}
  \sphinxDUC{2572}{\textbackslash}
\fi
\usepackage{cmap}
\usepackage[T1]{fontenc}
\usepackage{amsmath,amssymb,amstext}
\usepackage{babel}



\usepackage{times}
\expandafter\ifx\csname T@LGR\endcsname\relax
\else
% LGR was declared as font encoding
  \substitutefont{LGR}{\rmdefault}{cmr}
  \substitutefont{LGR}{\sfdefault}{cmss}
  \substitutefont{LGR}{\ttdefault}{cmtt}
\fi
\expandafter\ifx\csname T@X2\endcsname\relax
  \expandafter\ifx\csname T@T2A\endcsname\relax
  \else
  % T2A was declared as font encoding
    \substitutefont{T2A}{\rmdefault}{cmr}
    \substitutefont{T2A}{\sfdefault}{cmss}
    \substitutefont{T2A}{\ttdefault}{cmtt}
  \fi
\else
% X2 was declared as font encoding
  \substitutefont{X2}{\rmdefault}{cmr}
  \substitutefont{X2}{\sfdefault}{cmss}
  \substitutefont{X2}{\ttdefault}{cmtt}
\fi


\usepackage[Bjarne]{fncychap}
\usepackage{sphinx}

\fvset{fontsize=\small}
\usepackage{geometry}


% Include hyperref last.
\usepackage{hyperref}
% Fix anchor placement for figures with captions.
\usepackage{hypcap}% it must be loaded after hyperref.
% Set up styles of URL: it should be placed after hyperref.
\urlstyle{same}
\addto\captionsenglish{\renewcommand{\contentsname}{Contents:}}

\usepackage{sphinxmessages}
\setcounter{tocdepth}{1}



\title{LineageTree}
\date{Mar 10, 2020}
\release{}
\author{Leo Guignard}
\newcommand{\sphinxlogo}{\vbox{}}
\renewcommand{\releasename}{}
\makeindex
\begin{document}

\pagestyle{empty}
\sphinxmaketitle
\pagestyle{plain}
\sphinxtableofcontents
\pagestyle{normal}
\phantomsection\label{\detokenize{index::doc}}



\chapter{Modules}
\label{\detokenize{index:modules}}\index{lineageTree (class in LineageTree)@\spxentry{lineageTree}\spxextra{class in LineageTree}}

\begin{fulllineitems}
\phantomsection\label{\detokenize{index:LineageTree.lineageTree}}\pysiglinewithargsret{\sphinxbfcode{\sphinxupquote{class }}\sphinxcode{\sphinxupquote{LineageTree.}}\sphinxbfcode{\sphinxupquote{lineageTree}}}{\emph{file\_format=None}, \emph{tb=None}, \emph{te=None}, \emph{z\_mult=1.0}, \emph{mask=None}, \emph{file\_type=\textquotesingle{}\textquotesingle{}}, \emph{delim=\textquotesingle{}}, \emph{\textquotesingle{}}, \emph{eigen=False}}{}~\index{\_\_init\_\_() (LineageTree.lineageTree method)@\spxentry{\_\_init\_\_()}\spxextra{LineageTree.lineageTree method}}

\begin{fulllineitems}
\phantomsection\label{\detokenize{index:LineageTree.lineageTree.__init__}}\pysiglinewithargsret{\sphinxbfcode{\sphinxupquote{\_\_init\_\_}}}{\emph{file\_format=None}, \emph{tb=None}, \emph{te=None}, \emph{z\_mult=1.0}, \emph{mask=None}, \emph{file\_type=\textquotesingle{}\textquotesingle{}}, \emph{delim=\textquotesingle{}}, \emph{\textquotesingle{}}, \emph{eigen=False}}{}
Main library to build tree graph representation of lineage tree data
It can read TGMM, ASTEC, SVF, MaMuT and TrackMate outputs.
\begin{quote}\begin{description}
\item[{Parameters}] \leavevmode\begin{itemize}
\item {} 
\sphinxstyleliteralstrong{\sphinxupquote{file\_format}} (\sphinxstyleliteralemphasis{\sphinxupquote{string}}) \textendash{} either \sphinxhyphen{} path format to TGMM xmls
\sphinxhyphen{} path to the MaMuT xml
\sphinxhyphen{} path to the binary file

\item {} 
\sphinxstyleliteralstrong{\sphinxupquote{tb}} (\sphinxstyleliteralemphasis{\sphinxupquote{int}}) \textendash{} first time point (necessary for TGMM xmls only)

\item {} 
\sphinxstyleliteralstrong{\sphinxupquote{te}} (\sphinxstyleliteralemphasis{\sphinxupquote{int}}) \textendash{} last time point (necessary for TGMM xmls only)

\item {} 
\sphinxstyleliteralstrong{\sphinxupquote{z\_mult}} (\sphinxstyleliteralemphasis{\sphinxupquote{float}}) \textendash{} z aspect ratio if necessary (usually only for TGMM xmls)

\item {} 
\sphinxstyleliteralstrong{\sphinxupquote{mask}} (\sphinxstyleliteralemphasis{\sphinxupquote{SpatialImage}}) \textendash{} binary image that specify the region to read (for TGMM xmls only)

\item {} 
\sphinxstyleliteralstrong{\sphinxupquote{file\_type}} (\sphinxstyleliteralemphasis{\sphinxupquote{str}}) \textendash{} type of input file. Accepts:
‘TGMM, ‘ASTEC’, MaMuT’, ‘TrackMate’, ‘csv’, ‘celegans’, ‘binary’
default is ‘binary’

\end{itemize}

\end{description}\end{quote}

\end{fulllineitems}

\index{add\_node() (LineageTree.lineageTree method)@\spxentry{add\_node()}\spxextra{LineageTree.lineageTree method}}

\begin{fulllineitems}
\phantomsection\label{\detokenize{index:LineageTree.lineageTree.add_node}}\pysiglinewithargsret{\sphinxbfcode{\sphinxupquote{add\_node}}}{\emph{t=None}, \emph{succ=None}, \emph{pos=None}, \emph{id=None}, \emph{reverse=False}}{}
Adds a node to the lineageTree and update it accordingly.
\begin{quote}\begin{description}
\item[{Parameters}] \leavevmode\begin{itemize}
\item {} 
\sphinxstyleliteralstrong{\sphinxupquote{t}} (\sphinxstyleliteralemphasis{\sphinxupquote{int}}) \textendash{} int, time to which to add the node

\item {} 
\sphinxstyleliteralstrong{\sphinxupquote{succ}} (\sphinxstyleliteralemphasis{\sphinxupquote{int}}) \textendash{} id of the node the new node is a successor to

\item {} 
\sphinxstyleliteralstrong{\sphinxupquote{pos}} (\sphinxstyleliteralemphasis{\sphinxupquote{{[}}}\sphinxstyleliteralemphasis{\sphinxupquote{float}}\sphinxstyleliteralemphasis{\sphinxupquote{, }}\sphinxstyleliteralemphasis{\sphinxupquote{{]}}}) \textendash{} list of three floats representing the 3D spatial position of the node

\item {} 
\sphinxstyleliteralstrong{\sphinxupquote{id}} (\sphinxstyleliteralemphasis{\sphinxupquote{int}}) \textendash{} id value of the new node, to be used carefully,
if None is provided the new id is automatically computed.

\item {} 
\sphinxstyleliteralstrong{\sphinxupquote{reverse}} (\sphinxstyleliteralemphasis{\sphinxupquote{bool}}) \textendash{} True if in this lineageTree the predecessors are the successors and reciprocally.
This is there for bacward compatibility, should be left at False.

\end{itemize}

\item[{Returns}] \leavevmode
id of the new node.

\item[{Return type}] \leavevmode
int

\end{description}\end{quote}

\end{fulllineitems}

\index{compute\_k\_nearest\_neighbours() (LineageTree.lineageTree method)@\spxentry{compute\_k\_nearest\_neighbours()}\spxextra{LineageTree.lineageTree method}}

\begin{fulllineitems}
\phantomsection\label{\detokenize{index:LineageTree.lineageTree.compute_k_nearest_neighbours}}\pysiglinewithargsret{\sphinxbfcode{\sphinxupquote{compute\_k\_nearest\_neighbours}}}{\emph{k=10}}{}
Computes the k\sphinxhyphen{}nearest neighbors
Writes the output in the attribute \sphinxtitleref{kn\_graph}
and returns it.
\begin{quote}\begin{description}
\item[{Parameters}] \leavevmode
\sphinxstyleliteralstrong{\sphinxupquote{k}} (\sphinxstyleliteralemphasis{\sphinxupquote{float}}) \textendash{} number of nearest neighours

\item[{Returns}] \leavevmode
\begin{description}
\item[{dictionary that maps}] \leavevmode
a cell id to its \sphinxtitleref{k} nearest neighbors

\end{description}


\item[{Return type}] \leavevmode
\{int, set({[}int, ..{]})\}

\end{description}\end{quote}

\end{fulllineitems}

\index{compute\_spatial\_density() (LineageTree.lineageTree method)@\spxentry{compute\_spatial\_density()}\spxextra{LineageTree.lineageTree method}}

\begin{fulllineitems}
\phantomsection\label{\detokenize{index:LineageTree.lineageTree.compute_spatial_density}}\pysiglinewithargsret{\sphinxbfcode{\sphinxupquote{compute\_spatial\_density}}}{\emph{t\_b=None}, \emph{t\_e=None}, \emph{th=50}}{}
Computes the spatial density of cells between \sphinxtitleref{t\_b} and \sphinxtitleref{t\_e}.
The spatial density is computed as follow:
\#cell/(4/3*pi*th\textasciicircum{}3)
The results is stored in self.spatial\_density is returned.
\begin{quote}\begin{description}
\item[{Parameters}] \leavevmode\begin{itemize}
\item {} 
\sphinxstyleliteralstrong{\sphinxupquote{t\_b}} (\sphinxstyleliteralemphasis{\sphinxupquote{int}}) \textendash{} starting time to look at, default first time point

\item {} 
\sphinxstyleliteralstrong{\sphinxupquote{t\_e}} (\sphinxstyleliteralemphasis{\sphinxupquote{int}}) \textendash{} ending time to look at, default last time point

\item {} 
\sphinxstyleliteralstrong{\sphinxupquote{th}} (\sphinxstyleliteralemphasis{\sphinxupquote{float}}) \textendash{} size of the neighbourhood

\end{itemize}

\item[{Returns}] \leavevmode
dictionary that maps a cell id to its spatial density

\item[{Return type}] \leavevmode
\{int, float\}

\end{description}\end{quote}

\end{fulllineitems}

\index{compute\_spatial\_edges() (LineageTree.lineageTree method)@\spxentry{compute\_spatial\_edges()}\spxextra{LineageTree.lineageTree method}}

\begin{fulllineitems}
\phantomsection\label{\detokenize{index:LineageTree.lineageTree.compute_spatial_edges}}\pysiglinewithargsret{\sphinxbfcode{\sphinxupquote{compute\_spatial\_edges}}}{\emph{th=50}}{}
Computes the neighbors at a distance \sphinxtitleref{th}
Writes the output in the attribute \sphinxtitleref{th\_edge}
and returns it.
\begin{quote}\begin{description}
\item[{Parameters}] \leavevmode
\sphinxstyleliteralstrong{\sphinxupquote{th}} (\sphinxstyleliteralemphasis{\sphinxupquote{float}}) \textendash{} distance to consider neighbors

\item[{Returns}] \leavevmode
\begin{description}
\item[{dictionary that maps}] \leavevmode
a cell id to its neighbors at a distance \sphinxtitleref{th}

\end{description}


\item[{Return type}] \leavevmode
\{int, set({[}int, ..{]})\}

\end{description}\end{quote}

\end{fulllineitems}

\index{fuse\_nodes() (LineageTree.lineageTree method)@\spxentry{fuse\_nodes()}\spxextra{LineageTree.lineageTree method}}

\begin{fulllineitems}
\phantomsection\label{\detokenize{index:LineageTree.lineageTree.fuse_nodes}}\pysiglinewithargsret{\sphinxbfcode{\sphinxupquote{fuse\_nodes}}}{\emph{c1}, \emph{c2}}{}
Fuses together two nodes that belong to the same time point
and update the lineageTree accordingly.
\begin{quote}\begin{description}
\item[{Parameters}] \leavevmode\begin{itemize}
\item {} 
\sphinxstyleliteralstrong{\sphinxupquote{c1}} (\sphinxstyleliteralemphasis{\sphinxupquote{int}}) \textendash{} id of the first node to fuse

\item {} 
\sphinxstyleliteralstrong{\sphinxupquote{c2}} (\sphinxstyleliteralemphasis{\sphinxupquote{int}}) \textendash{} id of the second node to fuse

\end{itemize}

\end{description}\end{quote}

\end{fulllineitems}

\index{get\_cycle() (LineageTree.lineageTree method)@\spxentry{get\_cycle()}\spxextra{LineageTree.lineageTree method}}

\begin{fulllineitems}
\phantomsection\label{\detokenize{index:LineageTree.lineageTree.get_cycle}}\pysiglinewithargsret{\sphinxbfcode{\sphinxupquote{get\_cycle}}}{\emph{x}, \emph{depth=None}, \emph{depth\_pred=None}, \emph{depth\_succ=None}}{}
Computes the predecessors and successors of the node \sphinxstyleemphasis{x} up to
\sphinxstyleemphasis{depth\_pred} predecessors plus \sphinxstyleemphasis{depth\_succ} successors.
If the value \sphinxstyleemphasis{depth} is provided and not None,
\sphinxstyleemphasis{depth\_pred} and \sphinxstyleemphasis{depth\_succ} are overwrited by \sphinxstyleemphasis{depth}.
The ordered list of ids is returned.
\begin{quote}\begin{description}
\item[{Parameters}] \leavevmode\begin{itemize}
\item {} 
\sphinxstyleliteralstrong{\sphinxupquote{x}} (\sphinxstyleliteralemphasis{\sphinxupquote{int}}) \textendash{} id of the node to compute

\item {} 
\sphinxstyleliteralstrong{\sphinxupquote{depth}} (\sphinxstyleliteralemphasis{\sphinxupquote{int}}) \textendash{} maximum number of predecessors and successor to return

\item {} 
\sphinxstyleliteralstrong{\sphinxupquote{depth\_pred}} (\sphinxstyleliteralemphasis{\sphinxupquote{int}}) \textendash{} maximum number of predecessors to return

\item {} 
\sphinxstyleliteralstrong{\sphinxupquote{depth\_succ}} (\sphinxstyleliteralemphasis{\sphinxupquote{int}}) \textendash{} maximum number of successors to return

\end{itemize}

\item[{Returns}] \leavevmode
list of ids

\item[{Return type}] \leavevmode
{[}int, {]}

\end{description}\end{quote}

\end{fulllineitems}

\index{get\_gabriel\_graph() (LineageTree.lineageTree method)@\spxentry{get\_gabriel\_graph()}\spxextra{LineageTree.lineageTree method}}

\begin{fulllineitems}
\phantomsection\label{\detokenize{index:LineageTree.lineageTree.get_gabriel_graph}}\pysiglinewithargsret{\sphinxbfcode{\sphinxupquote{get\_gabriel\_graph}}}{\emph{t}}{}
Build the Gabriel graph of the given graph for time point \sphinxtitleref{t}
The Garbiel graph is then stored in self.Gabriel\_graph and returned
\sphinxstyleemphasis{WARNING: the graph is not recomputed if already computed. even if nodes were added}.
\begin{quote}\begin{description}
\item[{Parameters}] \leavevmode
\sphinxstyleliteralstrong{\sphinxupquote{t}} (\sphinxstyleliteralemphasis{\sphinxupquote{int}}) \textendash{} time

\item[{Returns}] \leavevmode
\begin{description}
\item[{a dictionary that maps a node to}] \leavevmode
the set of its neighbors

\end{description}


\item[{Return type}] \leavevmode
\{int, set({[}int, {]})\}

\end{description}\end{quote}

\end{fulllineitems}

\index{get\_idx3d() (LineageTree.lineageTree method)@\spxentry{get\_idx3d()}\spxextra{LineageTree.lineageTree method}}

\begin{fulllineitems}
\phantomsection\label{\detokenize{index:LineageTree.lineageTree.get_idx3d}}\pysiglinewithargsret{\sphinxbfcode{\sphinxupquote{get\_idx3d}}}{\emph{t}}{}
Get a 3d kdtree for the dataset at time \sphinxstyleemphasis{t} .
The  kdtree is stored in \sphinxstyleemphasis{self.kdtrees{[}t{]}}
\begin{quote}\begin{description}
\item[{Parameters}] \leavevmode
\sphinxstyleliteralstrong{\sphinxupquote{t}} (\sphinxstyleliteralemphasis{\sphinxupquote{int}}) \textendash{} time

\item[{Returns}] \leavevmode
\begin{description}
\item[{the built kdtree and}] \leavevmode
the correspondancy list,
If the query in the kdtree gives you the value i,
then it corresponds to the id in the tree to\_check\_self{[}i{]}

\end{description}


\item[{Return type}] \leavevmode
(kdtree, {[}int, {]})

\end{description}\end{quote}

\end{fulllineitems}

\index{get\_next\_id() (LineageTree.lineageTree method)@\spxentry{get\_next\_id()}\spxextra{LineageTree.lineageTree method}}

\begin{fulllineitems}
\phantomsection\label{\detokenize{index:LineageTree.lineageTree.get_next_id}}\pysiglinewithargsret{\sphinxbfcode{\sphinxupquote{get\_next\_id}}}{}{}
Computes the next authorized id.
\begin{quote}\begin{description}
\item[{Returns}] \leavevmode
next authorized id

\item[{Return type}] \leavevmode
int

\end{description}\end{quote}

\end{fulllineitems}

\index{get\_predecessors() (LineageTree.lineageTree method)@\spxentry{get\_predecessors()}\spxextra{LineageTree.lineageTree method}}

\begin{fulllineitems}
\phantomsection\label{\detokenize{index:LineageTree.lineageTree.get_predecessors}}\pysiglinewithargsret{\sphinxbfcode{\sphinxupquote{get\_predecessors}}}{\emph{x}, \emph{depth=None}}{}
Computes the predecessors of the node \sphinxstyleemphasis{x} up to
\sphinxstyleemphasis{depth} predecessors. The ordered list of ids is returned.
\begin{quote}\begin{description}
\item[{Parameters}] \leavevmode\begin{itemize}
\item {} 
\sphinxstyleliteralstrong{\sphinxupquote{x}} (\sphinxstyleliteralemphasis{\sphinxupquote{int}}) \textendash{} id of the node to compute

\item {} 
\sphinxstyleliteralstrong{\sphinxupquote{depth}} (\sphinxstyleliteralemphasis{\sphinxupquote{int}}) \textendash{} maximum number of predecessors to return

\end{itemize}

\item[{Returns}] \leavevmode
list of ids, the last id is \sphinxstyleemphasis{x}

\item[{Return type}] \leavevmode
{[}int, {]}

\end{description}\end{quote}

\end{fulllineitems}

\index{get\_sub\_tree() (LineageTree.lineageTree method)@\spxentry{get\_sub\_tree()}\spxextra{LineageTree.lineageTree method}}

\begin{fulllineitems}
\phantomsection\label{\detokenize{index:LineageTree.lineageTree.get_sub_tree}}\pysiglinewithargsret{\sphinxbfcode{\sphinxupquote{get\_sub\_tree}}}{\emph{x}, \emph{preorder=False}}{}
Computes the list of cells from the subtree spawned by \sphinxstyleemphasis{x}
The default output order is breadth first traversal.
Unless preorder is \sphinxtitleref{True} in that case the order is
Depth first traversal preordered.
\begin{quote}\begin{description}
\item[{Parameters}] \leavevmode\begin{itemize}
\item {} 
\sphinxstyleliteralstrong{\sphinxupquote{x}} (\sphinxstyleliteralemphasis{\sphinxupquote{int}}) \textendash{} id of root node

\item {} 
\sphinxstyleliteralstrong{\sphinxupquote{preorder}} (\sphinxstyleliteralemphasis{\sphinxupquote{bool}}) \textendash{} if True the output is preorder DFT

\end{itemize}

\item[{Returns}] \leavevmode
the ordered list of node ids

\item[{Return type}] \leavevmode
({[}int, ..{]})

\end{description}\end{quote}

\end{fulllineitems}

\index{get\_successors() (LineageTree.lineageTree method)@\spxentry{get\_successors()}\spxextra{LineageTree.lineageTree method}}

\begin{fulllineitems}
\phantomsection\label{\detokenize{index:LineageTree.lineageTree.get_successors}}\pysiglinewithargsret{\sphinxbfcode{\sphinxupquote{get\_successors}}}{\emph{x}, \emph{depth=None}}{}
Computes the successors of the node \sphinxstyleemphasis{x} up to
\sphinxstyleemphasis{depth} successors. The ordered list of ids is returned.
\begin{quote}\begin{description}
\item[{Parameters}] \leavevmode\begin{itemize}
\item {} 
\sphinxstyleliteralstrong{\sphinxupquote{x}} (\sphinxstyleliteralemphasis{\sphinxupquote{int}}) \textendash{} id of the node to compute

\item {} 
\sphinxstyleliteralstrong{\sphinxupquote{depth}} (\sphinxstyleliteralemphasis{\sphinxupquote{int}}) \textendash{} maximum number of predecessors to return

\end{itemize}

\item[{Returns}] \leavevmode
list of ids, the first id is \sphinxstyleemphasis{x}

\item[{Return type}] \leavevmode
{[}int, {]}

\end{description}\end{quote}

\end{fulllineitems}

\index{read\_from\_ASTEC() (LineageTree.lineageTree method)@\spxentry{read\_from\_ASTEC()}\spxextra{LineageTree.lineageTree method}}

\begin{fulllineitems}
\phantomsection\label{\detokenize{index:LineageTree.lineageTree.read_from_ASTEC}}\pysiglinewithargsret{\sphinxbfcode{\sphinxupquote{read\_from\_ASTEC}}}{\emph{file\_path}, \emph{eigen=False}}{}
Read an \sphinxtitleref{xml} or \sphinxtitleref{pkl} file produced by the ASTEC algorithm.
\begin{quote}\begin{description}
\item[{Parameters}] \leavevmode\begin{itemize}
\item {} 
\sphinxstyleliteralstrong{\sphinxupquote{file\_path}} (\sphinxstyleliteralemphasis{\sphinxupquote{str}}) \textendash{} path to an output generated by ASTEC

\item {} 
\sphinxstyleliteralstrong{\sphinxupquote{eigen}} (\sphinxstyleliteralemphasis{\sphinxupquote{bool}}) \textendash{} whether or not to read the eigen values, default False

\end{itemize}

\end{description}\end{quote}

\end{fulllineitems}

\index{read\_from\_binary() (LineageTree.lineageTree method)@\spxentry{read\_from\_binary()}\spxextra{LineageTree.lineageTree method}}

\begin{fulllineitems}
\phantomsection\label{\detokenize{index:LineageTree.lineageTree.read_from_binary}}\pysiglinewithargsret{\sphinxbfcode{\sphinxupquote{read\_from\_binary}}}{\emph{fname}, \emph{reverse\_time=False}}{}
Reads a binary lineageTree file name.
Format description: see self.to\_binary
\begin{quote}\begin{description}
\item[{Parameters}] \leavevmode\begin{itemize}
\item {} 
\sphinxstyleliteralstrong{\sphinxupquote{fname}} \textendash{} string, path to the binary file

\item {} 
\sphinxstyleliteralstrong{\sphinxupquote{reverse\_time}} \textendash{} bool, not used

\end{itemize}

\end{description}\end{quote}

\end{fulllineitems}

\index{read\_from\_csv() (LineageTree.lineageTree method)@\spxentry{read\_from\_csv()}\spxextra{LineageTree.lineageTree method}}

\begin{fulllineitems}
\phantomsection\label{\detokenize{index:LineageTree.lineageTree.read_from_csv}}\pysiglinewithargsret{\sphinxbfcode{\sphinxupquote{read\_from\_csv}}}{\emph{file\_path}, \emph{z\_mult}, \emph{link=1}, \emph{delim=\textquotesingle{}}, \emph{\textquotesingle{}}}{}
\end{fulllineitems}

\index{read\_from\_mamut\_xml() (LineageTree.lineageTree method)@\spxentry{read\_from\_mamut\_xml()}\spxextra{LineageTree.lineageTree method}}

\begin{fulllineitems}
\phantomsection\label{\detokenize{index:LineageTree.lineageTree.read_from_mamut_xml}}\pysiglinewithargsret{\sphinxbfcode{\sphinxupquote{read\_from\_mamut\_xml}}}{\emph{path}}{}
Read a lineage tree from a MaMuT xml.
\begin{quote}\begin{description}
\item[{Parameters}] \leavevmode
\sphinxstyleliteralstrong{\sphinxupquote{path}} (\sphinxstyleliteralemphasis{\sphinxupquote{str}}) \textendash{} path to the MaMut xml

\end{description}\end{quote}

\end{fulllineitems}

\index{read\_from\_txt\_for\_celegans() (LineageTree.lineageTree method)@\spxentry{read\_from\_txt\_for\_celegans()}\spxextra{LineageTree.lineageTree method}}

\begin{fulllineitems}
\phantomsection\label{\detokenize{index:LineageTree.lineageTree.read_from_txt_for_celegans}}\pysiglinewithargsret{\sphinxbfcode{\sphinxupquote{read\_from\_txt\_for\_celegans}}}{\emph{file}}{}
\end{fulllineitems}

\index{read\_tgmm\_xml() (LineageTree.lineageTree method)@\spxentry{read\_tgmm\_xml()}\spxextra{LineageTree.lineageTree method}}

\begin{fulllineitems}
\phantomsection\label{\detokenize{index:LineageTree.lineageTree.read_tgmm_xml}}\pysiglinewithargsret{\sphinxbfcode{\sphinxupquote{read\_tgmm\_xml}}}{\emph{file\_format}, \emph{tb}, \emph{te}, \emph{z\_mult=1.0}, \emph{mask=None}}{}
Reads a lineage tree from TGMM xml output.
\begin{quote}\begin{description}
\item[{Parameters}] \leavevmode\begin{itemize}
\item {} 
\sphinxstyleliteralstrong{\sphinxupquote{file\_format}} (\sphinxstyleliteralemphasis{\sphinxupquote{str}}) \textendash{} path to the xmls location.
it should be written as follow:
path/to/xml/standard\_name\_t\{t:06d\}.xml where (as an example)
\{t:06d\} means a series of 6 digits representing the time and
if the time values is smaller that 6 digits, the missing
digits are filed with 0s

\item {} 
\sphinxstyleliteralstrong{\sphinxupquote{tb}} (\sphinxstyleliteralemphasis{\sphinxupquote{int}}) \textendash{} first time point to read

\item {} 
\sphinxstyleliteralstrong{\sphinxupquote{te}} (\sphinxstyleliteralemphasis{\sphinxupquote{int}}) \textendash{} last time point to read

\item {} 
\sphinxstyleliteralstrong{\sphinxupquote{z\_mult}} (\sphinxstyleliteralemphasis{\sphinxupquote{float}}) \textendash{} aspect ratio

\item {} 
\sphinxstyleliteralstrong{\sphinxupquote{mask}} (\sphinxstyleliteralemphasis{\sphinxupquote{SpatialImage}}) \textendash{} binary image that specify the region to read

\end{itemize}

\end{description}\end{quote}

\end{fulllineitems}

\index{remove\_node() (LineageTree.lineageTree method)@\spxentry{remove\_node()}\spxextra{LineageTree.lineageTree method}}

\begin{fulllineitems}
\phantomsection\label{\detokenize{index:LineageTree.lineageTree.remove_node}}\pysiglinewithargsret{\sphinxbfcode{\sphinxupquote{remove\_node}}}{\emph{c}}{}
Removes a node and update the lineageTree accordingly
\begin{quote}\begin{description}
\item[{Parameters}] \leavevmode
\sphinxstyleliteralstrong{\sphinxupquote{c}} (\sphinxstyleliteralemphasis{\sphinxupquote{int}}) \textendash{} id of the node to remove

\end{description}\end{quote}

\end{fulllineitems}

\index{to\_binary() (LineageTree.lineageTree method)@\spxentry{to\_binary()}\spxextra{LineageTree.lineageTree method}}

\begin{fulllineitems}
\phantomsection\label{\detokenize{index:LineageTree.lineageTree.to_binary}}\pysiglinewithargsret{\sphinxbfcode{\sphinxupquote{to\_binary}}}{\emph{fname}, \emph{starting\_points=None}}{}
Writes the lineage tree (a forest) as a binary structure
(assuming it is a binary tree, it would not work for \sphinxstyleemphasis{n} ary tree with 2 \textless{} \sphinxstyleemphasis{n}).
The binary file is composed of 3 sequences of numbers and
a header specifying the size of each of these sequences.
The first sequence, \sphinxstyleemphasis{number\_sequence}, represents the lineage tree
as a DFT preporder transversal list. \sphinxhyphen{}1 signifying a leaf and \sphinxhyphen{}2 a branching
The second sequence, \sphinxstyleemphasis{time\_sequence}, represent the starting time of each tree.
The third sequence, \sphinxstyleemphasis{pos\_sequence}, reprensent the 3D coordinates of the objects.
The header specify the size of each of these sequences.
Each size is stored as a long long
The \sphinxstyleemphasis{number\_sequence} is stored as a list of long long (0 \sphinxhyphen{}\textgreater{} 2\textasciicircum{}(8*8)\sphinxhyphen{}1)
The \sphinxstyleemphasis{time\_sequence} is stored as a list of unsigned short (0 \sphinxhyphen{}\textgreater{} 2\textasciicircum{}(8*2)\sphinxhyphen{}1)
The \sphinxstyleemphasis{pos\_sequence} is stored as a list of double.
\begin{quote}\begin{description}
\item[{Parameters}] \leavevmode\begin{itemize}
\item {} 
\sphinxstyleliteralstrong{\sphinxupquote{fname}} (\sphinxstyleliteralemphasis{\sphinxupquote{str}}) \textendash{} name of the binary file

\item {} 
\sphinxstyleliteralstrong{\sphinxupquote{starting\_points}} (\sphinxstyleliteralemphasis{\sphinxupquote{{[}}}\sphinxstyleliteralemphasis{\sphinxupquote{int}}\sphinxstyleliteralemphasis{\sphinxupquote{, }}\sphinxstyleliteralemphasis{\sphinxupquote{{]}}}) \textendash{} list of the roots to be written.
If None, all roots are written, default value, None

\end{itemize}

\end{description}\end{quote}

\end{fulllineitems}

\index{to\_tlp() (LineageTree.lineageTree method)@\spxentry{to\_tlp()}\spxextra{LineageTree.lineageTree method}}

\begin{fulllineitems}
\phantomsection\label{\detokenize{index:LineageTree.lineageTree.to_tlp}}\pysiglinewithargsret{\sphinxbfcode{\sphinxupquote{to\_tlp}}}{\emph{fname}, \emph{t\_min=\sphinxhyphen{}1}, \emph{t\_max=inf}, \emph{nodes\_to\_use=None}, \emph{temporal=True}, \emph{spatial=None}, \emph{write\_layout=True}, \emph{node\_properties=None}, \emph{Names=False}}{}
Write a lineage tree into an understable tulip file.
\begin{quote}\begin{description}
\item[{Parameters}] \leavevmode\begin{itemize}
\item {} 
\sphinxstyleliteralstrong{\sphinxupquote{fname}} (\sphinxstyleliteralemphasis{\sphinxupquote{str}}) \textendash{} path to the tulip file to create

\item {} 
\sphinxstyleliteralstrong{\sphinxupquote{t\_min}} (\sphinxstyleliteralemphasis{\sphinxupquote{int}}) \textendash{} minimum time to consider, default \sphinxhyphen{}1

\item {} 
\sphinxstyleliteralstrong{\sphinxupquote{t\_max}} (\sphinxstyleliteralemphasis{\sphinxupquote{int}}) \textendash{} maximum time to consider, default np.inf

\item {} 
\sphinxstyleliteralstrong{\sphinxupquote{nodes\_to\_use}} (\sphinxstyleliteralemphasis{\sphinxupquote{{[}}}\sphinxstyleliteralemphasis{\sphinxupquote{int}}\sphinxstyleliteralemphasis{\sphinxupquote{, }}\sphinxstyleliteralemphasis{\sphinxupquote{{]}}}) \textendash{} list of nodes to show in the graph,
default \sphinxstyleemphasis{None}, then self.nodes is used
(taking into account \sphinxstyleemphasis{t\_min} and \sphinxstyleemphasis{t\_max})

\item {} 
\sphinxstyleliteralstrong{\sphinxupquote{temporal}} (\sphinxstyleliteralemphasis{\sphinxupquote{bool}}) \textendash{} True if the temporal links should be printed, default True

\item {} 
\sphinxstyleliteralstrong{\sphinxupquote{spatial}} (\sphinxstyleliteralemphasis{\sphinxupquote{str}}) \textendash{} Build spatial edges from a spatial neighbourhood graph.
The graph has to be computed before running this function
‘ball’: neighbours at a given distance,
‘kn’: k\sphinxhyphen{}nearest neighbours,
‘GG’: gabriel graph,
None: no spatial edges are writen.
Default None

\item {} 
\sphinxstyleliteralstrong{\sphinxupquote{write\_layout}} (\sphinxstyleliteralemphasis{\sphinxupquote{bool}}) \textendash{} True, write the spatial position as layout,
False, do not write spatial positionm
default True

\item {} 
\sphinxstyleliteralstrong{\sphinxupquote{node\_properties}} (\{\sphinxtitleref{p\_name}, {[}\{id, p\_value\}, default{]}\}) \textendash{} a dictionary of properties to write
To a key representing the name of the property is
paired a dictionary that maps a cell id to a property
and a default value for this property

\item {} 
\sphinxstyleliteralstrong{\sphinxupquote{Names}} (\sphinxstyleliteralemphasis{\sphinxupquote{bool}}) \textendash{} Only works with ASTEC outputs, True to sort the cells by their names

\end{itemize}

\end{description}\end{quote}

\end{fulllineitems}

\index{write\_to\_am() (LineageTree.lineageTree method)@\spxentry{write\_to\_am()}\spxextra{LineageTree.lineageTree method}}

\begin{fulllineitems}
\phantomsection\label{\detokenize{index:LineageTree.lineageTree.write_to_am}}\pysiglinewithargsret{\sphinxbfcode{\sphinxupquote{write\_to\_am}}}{\emph{path\_format}, \emph{t\_b=None}, \emph{t\_e=None}, \emph{length=5}, \emph{manual\_labels=None}, \emph{default\_label=5}, \emph{new\_pos=None}}{}
Writes a lineageTree into an Amira readable data (.am format).
\begin{quote}\begin{description}
\item[{Parameters}] \leavevmode\begin{itemize}
\item {} 
\sphinxstyleliteralstrong{\sphinxupquote{path\_format}} (\sphinxstyleliteralemphasis{\sphinxupquote{str}}) \textendash{} path to the output. It should contain 1 \%03d where the time step will be entered

\item {} 
\sphinxstyleliteralstrong{\sphinxupquote{t\_b}} (\sphinxstyleliteralemphasis{\sphinxupquote{int}}) \textendash{} first time point to write (if None, min(LT.to\_take\_time) is taken)

\item {} 
\sphinxstyleliteralstrong{\sphinxupquote{t\_e}} (\sphinxstyleliteralemphasis{\sphinxupquote{int}}) \textendash{} last time point to write (if None, max(LT.to\_take\_time) is taken)
note, if there is no ‘to\_take\_time’ attribute, self.time\_nodes
is considered instead (historical)

\item {} 
\sphinxstyleliteralstrong{\sphinxupquote{length}} (\sphinxstyleliteralemphasis{\sphinxupquote{int}}) \textendash{} length of the track to print (how many time before).

\item {} 
\sphinxstyleliteralstrong{\sphinxupquote{(}}\sphinxstyleliteralstrong{\sphinxupquote{\{id}} (\sphinxstyleliteralemphasis{\sphinxupquote{new\_pos}}) \textendash{} label, \}): dictionary that maps cell ids to

\item {} 
\sphinxstyleliteralstrong{\sphinxupquote{default\_label}} (\sphinxstyleliteralemphasis{\sphinxupquote{int}}) \textendash{} default value for the manual label

\item {} 
\sphinxstyleliteralstrong{\sphinxupquote{(}}\sphinxstyleliteralstrong{\sphinxupquote{\{id}} \textendash{} {[}x, y, z{]}\}): dictionary that maps a 3D position to a cell ID.
if new\_pos == None (default) then self.pos is considered.

\end{itemize}

\end{description}\end{quote}

\end{fulllineitems}

\index{write\_to\_svg() (LineageTree.lineageTree method)@\spxentry{write\_to\_svg()}\spxextra{LineageTree.lineageTree method}}

\begin{fulllineitems}
\phantomsection\label{\detokenize{index:LineageTree.lineageTree.write_to_svg}}\pysiglinewithargsret{\sphinxbfcode{\sphinxupquote{write\_to\_svg}}}{\emph{file\_name}, \emph{roots=None}, \emph{draw\_nodes=True}, \emph{draw\_edges=True}, \emph{order\_key=None}, \emph{vert\_space\_factor=0.5}, \emph{horizontal\_space=1}, \emph{node\_size=None}, \emph{stroke\_width=None}, \emph{factor=1.0}, \emph{node\_color=None}, \emph{stroke\_color=None}, \emph{positions=None}}{}
Writes the lineage tree to an SVG file.
Node and edges coloring and size can be provided.
\begin{quote}\begin{description}
\item[{Parameters}] \leavevmode\begin{itemize}
\item {} 
\sphinxstyleliteralstrong{\sphinxupquote{file\_name}} \textendash{} str, filesystem filename valid for \sphinxtitleref{open()}

\item {} 
\sphinxstyleliteralstrong{\sphinxupquote{roots}} \textendash{} {[}int, …{]}, list of node ids to be drawn. If \sphinxtitleref{None} all the nodes will be drawn. Default \sphinxtitleref{None}

\item {} 
\sphinxstyleliteralstrong{\sphinxupquote{draw\_nodes}} \textendash{} bool, wether to print the nodes or not, default \sphinxtitleref{True}

\item {} 
\sphinxstyleliteralstrong{\sphinxupquote{draw\_edges}} \textendash{} bool, wether to print the edges or not, default \sphinxtitleref{True}

\item {} 
\sphinxstyleliteralstrong{\sphinxupquote{order\_key}} \textendash{} function that would work for the attribute \sphinxtitleref{key=} for the \sphinxtitleref{sort}/\sphinxtitleref{sorted} function

\item {} 
\sphinxstyleliteralstrong{\sphinxupquote{vert\_space\_factor}} \textendash{} float, the vertical position of a node is its time. \sphinxtitleref{vert\_space\_factor} is a
multiplier to space more or less nodes in time

\item {} 
\sphinxstyleliteralstrong{\sphinxupquote{horizontal\_space}} \textendash{} float, space between two consecutive nodes

\item {} 
\sphinxstyleliteralstrong{\sphinxupquote{node\_size}} \textendash{} func, a function that maps a node id to a \sphinxtitleref{float} value that will determine the
radius of the node. The default function return the constant value \sphinxtitleref{vertical\_space\_factor/2.1}

\item {} 
\sphinxstyleliteralstrong{\sphinxupquote{stroke\_width}} \textendash{} func, a function that maps a node id to a \sphinxtitleref{float} value that will determine the
width of the daughter edge.  The default function return the constant value \sphinxtitleref{vertical\_space\_factor/2.1}

\item {} 
\sphinxstyleliteralstrong{\sphinxupquote{factor}} \textendash{} float, scaling factor for nodes positions, default 1

\item {} 
\sphinxstyleliteralstrong{\sphinxupquote{node\_color}} \textendash{} func, a function that maps a node id to a triplet between 0 and 255.
The triplet will determine the color of the node.

\item {} 
\sphinxstyleliteralstrong{\sphinxupquote{stroke\_color}} \textendash{} func, a function that maps a node id to a triplet between 0 and 255.
The triplet will determine the color of the stroke of the inward edge.

\item {} 
\sphinxstyleliteralstrong{\sphinxupquote{positions}} \textendash{} \{int: {[}float, float{]}, …\}, dictionary that maps a node id to a 2D position.
Default \sphinxtitleref{None}. If provided it will be used to position the nodes.

\end{itemize}

\end{description}\end{quote}

\end{fulllineitems}


\end{fulllineitems}




\renewcommand{\indexname}{Index}
\printindex
\end{document}